%!TEX root = paper.tex
\section{Overview}\label{sec-overview}

\gls{oin}, Vienna's community-operated \gls{IoT} network, hinges on
three architectural elements supporting its operation:
First, the wireless connectivity between sensors and the network
uses \gls{lora} and \gls{lorawan} specifically.
Second, the gateways and backend operate according to the architecture
proposed by \gls{ttn}~\cite{ttn}.
Third, gateways, backend, Internet connectivity etc. are built, operated,
maintained, and contributed by volunteer community members of \gls{oin},
all under the premise that the system is open to the public for use and
contributions on all layers.
In the following, we describe each aspect in greater detail.


\subsection{\gls{lora} and \gls{lorawan}}

\gls{lora} is a proprietary layer-1 technology for low-power long-range
low-bandwidth wireless transmissions. It operates in license-exempt
frequency ranges (the 433 MHz (EMEA, northern Asia) and 915 MHz
(the Americas) \acrshort{ISM} bands, and the European 868 MHz
\acrshort{SRD} band).



.... tech for l1/l2 ....



\subsection{\gls{ttn}'s system architecture}


\subsection{The \gls{oin} community}
