%!TEX root = paper.tex
\section{Overview}\label{sec-overview}

\gls{oin}, Vienna's community-operated \gls{IoT} network, hinges on
three architectural elements supporting its operation:
First, the wireless connectivity between sensors and the network
uses \gls{lora} and \gls{lorawan} specifically.
Second, the gateways and backend operate according to the architecture
proposed by \gls{ttn}~\cite{ttn}.
Third, gateways, backend, Internet connectivity etc. are built, operated,
maintained, and contributed by volunteer community members of \gls{oin},
all under the premise that the system is open to the public for use and
contributions on all layers.
In the following, we describe each aspect in greater detail.


\subsection{\gls{lora} and \gls{lorawan}}

\gls{lora} is a proprietary layer-1 technology for low-power long-range
low-bandwidth wireless transmissions. It operates in license-exempt
frequency ranges (the 433 MHz (EMEA, northern Asia) and 915 MHz
(the Americas) \acrshort{ISM} bands, and the European 868 MHz
\acrshort{SRD} band). The transmit power is thus limited to around
10 dBm (10 milliwatts), depending on local legislation.
\gls{lora} symbols consist of constant-amplitude up- and downchirps,
with an overall configurable chirp rate (called ``spreading factor'')
and bandwidth.
A certain degree of orthogonality between temporally overlapping
transmissions is achieved by using different sub-ranges of
the current frequency band (better), and assigning different spreading
factors to different senders (less optimal)~\cite{croce}.
Otherwise, senders are assumed to listen-before-talk, use low
transmit power, and limit their active duty cycles explicitly
according to radio regulations in the frequency band, and implicitly
due to limited available (battery) power.
Depending on the actual modulation parameters used, \gls{lora}
achieves bit rates of 300-21,000 bits per second, and message lengths
of 20 to 250 bytes.
% Sensitivity -148 dBm, https://www.microchip.com/design-centers/wireless-connectivity/low-power-wide-area-networks/lora-technology
%
% Message length on the air 50-1500 ms

\gls{lorawan}~\cite{lorawan-specs} builds on \gls{lora} and adds
a message format usable in a system with wireless end devices,
access gateways, and network servers for further data processing.
The message format
supports two layers of AES-128 cryptography to secure
the message contents. A Network Session Key is used to calculate a
message integrity code, and to encrypt MAC-only (i.e., not
application-oriented) messages. The Application Session Key encrypts
application-specific messages. Both session keys may derived in an
over-the-air activation procedure that assumes a pre-shared
Application Identifier (AppEUI) to be present in the device.
In addition, the device's globally unique EUI-64 address is used
for session key derivation. Thus, every device derives its own
application- and device-specific Network and Application session key,
and key compromise only affects that single device.
Besides, the message format supports framing information, managing
endpoint addressing, acknowledgement requests and retransmissions,
data rate adaptations, etc.


\subsection{\gls{ttn}'s system architecture}


\subsection{The \gls{oin} community}
