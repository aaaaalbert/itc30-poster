
%% This file is based on bare_conf.tex
%% V1.4b
%% 2015/08/26
%% by Michael Shell
%% See:
%% http://www.michaelshell.org/
%% for current contact information.



\documentclass[conference]{IEEEtran}
\usepackage[utf8]{inputenc}
\usepackage{cite}
\usepackage{graphicx}
\usepackage{url}
\usepackage{todonotes}
%\usepackage{hyperref}
\usepackage{flushend}

\usepackage{glossaries}
\loadglsentries{acronyms}

% correct bad hyphenation here
\hyphenation{op-tical net-works semi-conduc-tor}


\begin{document}

\title{\gls{oin}: A Free, Open Community \acrshort{IoT} Network based on \gls{lorawan}}


% author names and affiliations
% use a multiple column layout for up to three different
% affiliations
\author{\IEEEauthorblockN{Albert}
\IEEEauthorblockA{}
\and
\IEEEauthorblockN{Stefan}
\IEEEauthorblockA{}
}


% make the title area
\maketitle

%!TEX root = paper.tex
\begin{abstract}
The abstract goes here.
\end{abstract}


%!TEX root = paper.tex
\section{Introduction}


%!TEX root = paper.tex
\section{Overview}\label{sec-overview}

\gls{oin}, Vienna's community-operated \gls{IoT} network, hinges on
three architectural elements supporting its operation:
First, the wireless connectivity between sensors and the network
uses \gls{lora} and \gls{lorawan} specifically.
Second, the gateways and backend operate according to the architecture
proposed by \gls{ttn}~\cite{ttn}.
Third, gateways, backend, Internet connectivity etc. are built, operated,
maintained, and contributed by volunteer community members of \gls{oin},
all under the premise that the system is open to the public for use and
contributions on all layers.
In the following, we describe each aspect in greater detail.


\subsection{\gls{lora} and \gls{lorawan}}

\gls{lora} is a proprietary layer-1 technology for low-power long-range
low-bandwidth wireless transmissions. It operates in license-exempt
frequency ranges (the 433 MHz (EMEA, northern Asia) and 915 MHz
(the Americas) \acrshort{ISM} bands, and the European 868 MHz
\acrshort{SRD} band).



.... tech for l1/l2 ....



\subsection{\gls{ttn}'s system architecture}


\subsection{The \gls{oin} community}

%!TEX root = paper.tex
\section*{Acknowledgements}

openiot.network has been partially supported by ....

The first author has been supported by ....

The authors thank the openiot.network community.





% For peer review papers, you can put extra information on the cover
% page as needed:
% \ifCLASSOPTIONpeerreview
% \begin{center} \bfseries EDICS Category: 3-BBND \end{center}
% \fi
%
% For peerreview papers, this IEEEtran command inserts a page break and
% creates the second title. It will be ignored for other modes.
\IEEEpeerreviewmaketitle

%\bibliography{iot}
%\bibliographystyle{plain}


\end{document}

