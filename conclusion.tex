%!TEX root = paper.tex
\section{Conclusion and Future Work}\label{sec:conclusion}

This paper presents a first look at \gls{lorawan} traffic
characteristics in the \gls{oin} community \gls{IoT} network
in Vienna, Austria.
The system is open for participation and use, which is interesting
for two main aspects: First, it enables use cases beyond traditional
centralized network operations and paid-for access.
Second, it fosters collaboration both within the system and across
its boundaries, this paper serving as an example for the latter.
Opportunities to get involved are plentiful for academia, and
prospects of knowledge gains abound.

The aggregate traffic characteristics presented show that \gls{oin}
currently handles 70 to 120 thousand messages per day, by far most
of which are shorter than 50 bytes. The most active gateway out of
13 in the Vienna region sees about a quarter of the traffic.
\\

The system used for aggregated statistics in this paper has its
limitations, and we are actively working on overcoming them.
For example, it currently lacks ways to investigate message
inter-arrival times, making it difficult to discuss models for
arrival processes.

Our ongoing efforts also include an anonymization
layer for message contents~\cite{lora-gateway-anonymize}.
Using it, we will be able to evaluate more detailed information
about received messages (including sender addresses, application
port numbers, retransmission counters etc.) in a privacy-respecting
fashion. It will also allow us to investigate spectrum usage
and traffic aggregation at various levels of hierarchy across
the \gls{oin} system.

Furthermore, we presented a ``passive'', gateway-centric view.
In the future, we plan to corroborate our studies using active
mapping with senders that we control. We note however that individual
efforts in this direction are already being carried out by \gls{oin}
participants, e.g. with bicycle-mounted \gls{lorawan} \acrshort{GPS}
beacons.
